\title{Meta-programming in OCaml\\ \large{Ppxlib: How We Got Here and\\Where We Are Now}}

\author{Paul-Elliot Anglès d'Auriac, Carl Eastlund, Sonja Heinze}

\newcommand{\abstractText}{\noindent
About nine years ago, the OCaml community let go of the "maintenance hell" of
source code meta-programming (Camlp4/5). Since then, meta-programming is mostly
done on the parsetree and has incrementally evolved into a solid ecosystem,
enriching the OCaml developer experience. Two highlights of that evolvement:
introducing parsetree migrations to gain individual cross-compiler-version compatibility
(OMP); and orchestrating those migrations to create an up-to-date ecosystem with
clear composition semantics and good performance (ppxlib).

In this talk, we outline the mentioned evolvement and explain the remaining
challenges, i.e. creating a good higienic meta-programming culture and offering
the best in terms of stability. }


%%%%%%%%%%%%%%%%%
% Configuration %
%%%%%%%%%%%%%%%%%

\documentclass[10pt, a4paper, twocolumn]{article}
\usepackage{xurl}
\usepackage[super,comma,sort&compress]{natbib}
\usepackage{abstract}
\renewcommand{\abstractnamefont}{\normalfont\bfseries}
\renewcommand{\abstracttextfont}{\normalfont\small\itshape}
\usepackage{amsmath}
\usepackage[framemethod=TikZ]{mdframed}
\usepackage{hyperref}

\newcounter{example}[section]\setcounter{example}{0}
\renewcommand{\theexample}{\arabic{section}.\arabic{example}}
\newenvironment{example}[2][]{%
%\refstepcounter{example}%
\ifstrempty{#1}%
{\mdfsetup{%
frametitle={%
\tikz[baseline=(current bounding box.east),outer sep=0pt]
\node[anchor=east,rectangle,fill=blue!20]
{\strut Example~\theexample};}}
}%
{\mdfsetup{%
frametitle={%
\tikz[baseline=(current bounding box.east),outer sep=0pt]
\node[anchor=east,rectangle,fill=blue!20]
{\strut Example:~#1};}}%
}%
\mdfsetup{innertopmargin=10pt,linecolor=blue!20,%
linewidth=2pt,topline=true,%
frametitleaboveskip=\dimexpr-\ht\strutbox\relax
}
\begin{mdframed}[]\relax%
\label{#2}}{\end{mdframed}}


%%%%%%%%%%%%%%
% References %
%%%%%%%%%%%%%%

% If changing the name of the bib file, change \bibliography{test} at the bottom
\begin{filecontents}{references.bib}

\end{filecontents}

% Any configuration that should be done before the end of the preamble:
\usepackage{hyperref}
\hypersetup{colorlinks=true, urlcolor=blue, linkcolor=blue, citecolor=blue}

\begin{document}

%%%%%%%%%%%%
% Abstract %
%%%%%%%%%%%%

\twocolumn[
  \begin{@twocolumnfalse}
    \maketitle
    \begin{abstract}
      \abstractText
      \newline
      \newline
    \end{abstract}
  \end{@twocolumnfalse}
]

%%%%%%%%%%%
% Article %
%%%%%%%%%%%

\section{Introduction}
Meta-programming enhances the programming language experience by opening the
door to features such as boilerplate generation, conditional compilation,
domain-specific syntax extensions, or generated file inclusion. It can be done
on any representation of the program.

As a statically typed language, OCaml successively constructs two in-memory
representations of the raw representation, i.e. the source-code.

\begin{align*}
source \xrightarrow{\text{parsing}} parsetree \xrightarrow{\text{typing}} typedtree
\end{align*}

Nowadays, the clear standard for OCaml meta-programming is parsetree focused and
is done via a compiler-external package called $ppxlib$. As of today, $\sim2000$
OCaml packages out of $\sim4200$ OCaml packages in total (on $opam$) depend
transitively on $ppxlib$.

\section{Historic overview}

On the way to that standard, we've had a long history of OCaml meta-programming
facilities.


\subsection{Source-code based}

The initial approach was source-code focused. It worked by extending the parser
to one's needs \cite{camlp4}:

\begin{align*}
source \xrightarrow[\text{parsing}]{\text{extended}} parsetree \xrightarrow{\text{typing}} typedtree
\end{align*}

Among many other problems, maintaining that parallel parser with extensible
grammar was a huge maintenance burden.


\subsection{Parsetree based}

That's why in 2014 the approach became parsetree (\emph{AST} from now on)
focused with $ocaml.4.02.00$. The AST was augmented by two meta-programming
dedicated nodes \cite{extension_nodes}\cite{attributes}. Meta-programming tools,
called PPXs, extend those nodes to other nodes:

\begin{gather*}
source \xrightarrow{\text{parsing}} AST \xrightarrow{\text{PPXs}}\\
\rightarrow expanded\ AST\xrightarrow{\text{typing}} typedtree
\end{gather*}

To extend the nodes, PPXs need to manipulate the AST data type. Changes to the
AST data type would often break PPXes that depended on a previous form.
% ceastlund: I don't know of earlier workarounds. I don't think _every_ ppx broke on
% _every_ version, but the problem accelerated over time.

To improve cross-compiler-version compatibility, $ocaml\mbox{-}migrate\mbox{-}parsetree$
(OMP)\cite{omp} was invented. It converts the compiler AST version to the
version used by the PPX, then applies the PPX, and then converts back. However,
downward conversions are only well-defined if the AST doesn't contain new
features/nodes. So full cross-compiler-version compatibility is only given if
\begin{align*}
\text{PPX\ AST\ version} >= \text{compiler\ AST\ version}
\end{align*}
That meant that the OCaml PPX world started to lag behind the compiler releases.

Initially, each PPX was a separate binary. The one early exception was $ppx \_
deriving$, which orchestrates multiple type-based derivers into a single
rewriter.

In 2018, $ppxlib$\cite{ppxlib} was created, in part to solve these problems. It
exposes a single version of the AST, which is kept up to date with the latest
compiler. This consolidates
the PPX ecosystem and upgrades most PPXes ``for free''.

The primary remaining problem is the few PPXes that get broken each time the AST
changes. Our initial proposal, presented at the OCaml workshop
2019\cite{workshop2019}, was to create a single AST that could contain fragments
from multiple versions at once. This proved too complex and was subsequently
abandoned. We now approach \hyperref[stability]{stability} with a combination of
technical and social solutions.

\subsection{Typedtree based}

There have also been efforts to introduce typedtree based meta-programming,
especially $MetaOCaml$\cite{metaocaml}. MetaOCaml and PPXes are different
notions of metaprogramming with different goals and expressivity. MetaOCaml is
type-safe, staged metaprogramming that can be used to specialize values at
compile-time or run-time. PPXes offer more general code generation than
specializing values, but lack MetaOCaml's ability to evaluate the code and its
integration with the type system. MetaOCaml requires non-merged changes
to the compiler and therefore only works on a fork of OCaml.

\section{Current Situation}

PPX machinery and the $ppxlib$ library have become the standard tool for OCaml
meta-programming. They allow arbitrary AST rewriting in general, and streamlined
special cases for context-free rewriting at extension nodes as well as derived
code at types and module type definitions. The deriver links together multiple
rewriters into one binary, executing all derivers and context-free rewriting in
a single pass over an AST.

\vspace{20pt}

(visualization)

\vspace{25pt}

Context-free transformations have three big advantages over more general rewriting:
\begin{enumerate}

\item\hypertarget{benefits} They compose well together.

\item Running in a single pass over a file is a significant performance win.

\item\label{responsibility} They are easier to reason-about than transformations
that might affect, or be affected by, their context.

\end{enumerate}

\subsection{Hygiene}

Point \ref{responsibility}. above touches on an important topic: PPXs generate
the final representation of the program and so have an impact on how the
compiler and the editor tool $merlin$ analyze the program. That's why we define
clear hygiene rules\cite{hygiene} and try to create a culture in which PPX
authors are aware of the possible consequences. Some of these hygiene rules are
enforced technically by $ppxlib$, and others are currently only guidelines that
PPX authors should follow. The PPX mechanism within the compiler itself does not
enforce any technical notion of hygiene as in some macro
systems\cite{scheme_hygiene}.

\subsubsection*{Error handling}

Errors in OCaml are most often handled by raising exceptions. Naive error
handling within the $ppxlib$ driver would abort transformation of an entire
program the first time an exception was raised. This means tools like $merlin$
learn nothing about the shape of the transformed program, and debugging the
problem becomes only more difficult.

Our first approach to addressing this was to provide tools for expressing errors
as AST nodes, and encouraging PPX authors to use these instead of exceptions.
This practice was slow to creep into the ecosystem, so we made the behavior
automatic by handling exceptions inside $ppxlib$. Now an exception raised by a
context-free transformation is caught, and its error message is stored as an AST
node so that the rest of the program can be transformed.

\subsubsection*{Location invariants}

The \emph{Location} of a AST node reveals which part of the source code the node
corresponds to. We've run into some problems with transformed code, wherein
tools like $merlin$ cannot figure out which piece of transformed or duplicated
code to analyze when the user queries the source program.
%Locations are very important for the compiler/$merlin$ to point to the right
%place when reporting errors/warnings.

To resolve this issue, we've instituted two location invariants for the results
of PPX transformations. These can be configured in the $ppxlib$ driver as
automatic checks, or left to PPX authors to enforce.
\begin{itemize}
\item The locations of child nodes are contained by the locations of their parents.
\item The locations of sibling nodes don't overlap.
\end{itemize}

\subsubsection*{Full qualification of identifiers and operators}

A PPX needs to be independent of the semantic context it's applied in. This
property has no enforcement; it is the job of PPX authors to observe it so that
their transformations are robust.

\begin{example}[Non-qualified identifier]{example:non-qualified}
Suppose you have a PPX which injects code containing
\begin{verbatim}
compare x y
\end{verbatim}
Then the result of that comparison depends on
whether \texttt{Stdlib}'s \texttt{compare} is shadowed inside the context the
PPX is applied in or not.
\end{example}
\begin{example}[Fully qualified identifier]{example:non-qualified}
Now suppose your PPX instead fully qualifies:
\begin{verbatim}
Stdlib.compare x y
\end{verbatim}
Then the result is deterministic (we assume people don't shadow
the \texttt{Stdlib}).
\end{example}

If your PPX relies on definitions that don't form part of the \texttt{Stdlib},
the standard practice is to mint a runtime module for your PPX, give it an
explicit name such as \texttt{Ppx\_example\_runtime\_lib}.

This property can be selectively ignored to create a deliberately non-hygienic
interface. For example, the $ppx \_ expect$ PPX uses the name
\texttt{Expect\_test\_config} non-hygienically. Users may shadow this module
name to reconfigure expect tests.

%\subsubsection{Quoting of injected values}
%
%(What do you think Carl and Paul-Elliot? Should we also mention quoting here?)

\subsection{Stability}\label{stability}

Additionally to being hygienic, we also want our PPX ecosystem to stay
consolidated and up-to-date. The problem: when a new compiler with its new AST
version is released and $ppxlib$ bumps its exposed AST to the new version, there
are a few PPXs that break. We have two strategies here.

\subsubsection*{Reduce the number of breakages}

There are different ways to (de-)construct the AST. A very low-level approach is
to handle the original data type directly. That workflow is clearly unstable by
the unstable AST nature. A very high-level approach is to use the PPX
$metaquot$, which comes bundled with $ppxlib$. It lets you (de-)construct nodes
by writing the corresponding OCaml syntax. That's stable due to the OCaml syntax
being backward compatible. An intermediate approach is to use helper modules
called \texttt{Ast\_builder} and \texttt{Ast\_pattern}. That used to be unstable
since the modules are automatically generated. Good news: since $ppxlib.0.26.0$
(2022), we keep them manually stable!

%helper modules called $Ast\_builder$ and $Ast\_pattern$
%
%, such as:
%\begin{enumerate}
%\item\label{direct}
%Very low-level: directly handling the original data type.
%\item\label{helpers}
%More high-level: using helper modules called $Ast\_builder$ and $Ast\_pattern$.
%\item\label{metaquot}
%High-level: using the PPX $metaquot$ that comes bundled with $ppxlib$ and
%(de-)constructs nodes by writing original OCaml syntax.
%\end{enumerate}
%
%\ref{direct}. is unstable by the unstable parsetree nature. \ref{metaquot}. is stable due to
%the OCaml syntax being backward compatible. And \ref{helpers}. used to be
%unstable due to being automatically generated; however, since $ppxlib.0.26.0 $
%(2022), we keep them manually stable. When possible, we recommend
%avoiding \ref{direct}.

\subsubsection*{Patch each PPX in case of breakage}

Even with that, there are still a few PPXs that break. Therefore, when bumping
the AST, we create a big workspace containing all PPXs released on $opam$
(fulfilling a few standards), called the $ppx\_universe$\cite{universe}. We then
quickly patch the broken PPXs in the universe and open PRs. The few affected PPX
maintainers only need to review, merge and release.

%%%%%%%%%%%%%%
% References %
%%%%%%%%%%%%%%

\nocite{*}
\bibliographystyle{plain}
\bibliography{references}

\end{document}

\documentclass[12pt, a4paper, twocolumn]{article}
\usepackage{xurl}
\usepackage[super,comma,sort&compress]{natbib}
\usepackage{abstract}
\renewcommand{\abstractnamefont}{\normalfont\bfseries}
\renewcommand{\abstracttextfont}{\normalfont\small\itshape}
\usepackage{lipsum}

%%%%%%%%%%%%%%
% References %
%%%%%%%%%%%%%%

% If changing the name of the bib file, change \bibliography{test} at the bottom
\begin{filecontents}{references.bib}

@misc{camlp4,
  title        = "Link Title",
  author       = "Link Creator(s)",
  howpublished = "\url{https://example.com/}",
}

@misc{Author1,
  author       = "LastName, FirstName",
  howpublished = "\url{mailto:email@example.com}",
}

@article{ArticleReference1,
  author  = "Lastname1, Firstname1 and Lastname2, Firstname2",
  title   = "Article title",
  year    = "Year",
  journal = "Journal name",
  note    = "\url{https://dx.doi.org/...}",
}

\end{filecontents}

% Any configuration that should be done before the end of the preamble:
\usepackage{hyperref}
\hypersetup{colorlinks=true, urlcolor=blue, linkcolor=blue, citecolor=blue}

\nocite{*}
\bibliographystyle{plain}
\bibliography{test}

\end{document}
