\title{Meta-programming in OCaml\\ \large{Ppxlib: How We Got Here and\\Where We Are Now}}

\author{Paul-Elliot Anglès d'Auriac (?), Carl Eastlund, Sonja Heinze}

\newcommand{\abstractText}{\noindent
About nine years ago, the OCaml community let go of the "maintenance hell" of
source code meta-programming (Camlp4/5). Since then, meta-programming is mostly
done on the parsetree and has incrementally evolved into a solid ecosystem,
enriching the OCaml developer experience. Two highlights of that evolvement:
introducing parsetree migrations to gain individual cross-compiler compatibility
(OMP); and orchestrating those migrations to create an up-to-date ecosystem with
clear composition semantics and good performance (ppxlib).

In this talk, we outline the mentioned evolvement and explain the remaining
challenges, i.e. creating a good higienic meta-programming culture and offering
the best in terms of stability. }


%%%%%%%%%%%%%%%%%
% Configuration %
%%%%%%%%%%%%%%%%%

\documentclass[10pt, a4paper, twocolumn]{article}
\usepackage{xurl}
\usepackage[super,comma,sort&compress]{natbib}
\usepackage{abstract}
\renewcommand{\abstractnamefont}{\normalfont\bfseries}
\renewcommand{\abstracttextfont}{\normalfont\small\itshape}
\usepackage{amsmath}
\usepackage[framemethod=TikZ]{mdframed}
\usepackage{hyperref}

\newcounter{example}[section]\setcounter{example}{0}
\renewcommand{\theexample}{\arabic{section}.\arabic{example}}
\newenvironment{example}[2][]{%
%\refstepcounter{example}%
\ifstrempty{#1}%
{\mdfsetup{%
frametitle={%
\tikz[baseline=(current bounding box.east),outer sep=0pt]
\node[anchor=east,rectangle,fill=blue!20]
{\strut Example~\theexample};}}
}%
{\mdfsetup{%
frametitle={%
\tikz[baseline=(current bounding box.east),outer sep=0pt]
\node[anchor=east,rectangle,fill=blue!20]
{\strut Example:~#1};}}%
}%
\mdfsetup{innertopmargin=10pt,linecolor=blue!20,%
linewidth=2pt,topline=true,%
frametitleaboveskip=\dimexpr-\ht\strutbox\relax
}
\begin{mdframed}[]\relax%
\label{#2}}{\end{mdframed}}


%%%%%%%%%%%%%%
% References %
%%%%%%%%%%%%%%

% If changing the name of the bib file, change \bibliography{test} at the bottom
\begin{filecontents}{references.bib}

\end{filecontents}

% Any configuration that should be done before the end of the preamble:
\usepackage{hyperref}
\hypersetup{colorlinks=true, urlcolor=blue, linkcolor=blue, citecolor=blue}

\begin{document}

%%%%%%%%%%%%
% Abstract %
%%%%%%%%%%%%

\twocolumn[
  \begin{@twocolumnfalse}
    \maketitle
    \begin{abstract}
      \abstractText
      \newline
      \newline
    \end{abstract}
  \end{@twocolumnfalse}
]

%%%%%%%%%%%
% Article %
%%%%%%%%%%%

\section{Introduction}
Meta-programming enhances the programming language experience by opening the
door to features such as boilerplate generation, conditional compilation,
domain-specific syntax extensions, or generated file inclusion. It can be done
on any representation of the program.

As a statically typed language, OCaml successively constructs two in-memory
representations of the raw representation, i.e. the source-code.

\begin{align*}
source \xrightarrow{\text{parsing}} parsetree \xrightarrow{\text{typing}} typedtree
\end{align*}

Nowadays, the clear standard for OCaml meta-programming is parsetree focused and
is done via a compiler-external package called $ppxlib$. As of today, $\sim2000$
OCaml packages out of $\sim4200$ OCaml packages in total (on $opam$) depend
transitively on $ppxlib$.

\section{Historic overview}

On the way to that standard, we've had a long history of OCaml meta-programming
facilities.


\subsection{Source-code based}

The initial approach was source-code focused. It worked by extending the parser
to one's needs \cite{camlp4}:

\begin{align*}
source \xrightarrow[\text{parsing}]{\text{extended}} parsetree \xrightarrow{\text{typing}} typedtree
\end{align*}

Among many other problems, maintaining that parallel parser with extensible
grammar was a huge maintenance burden.


\subsection{Parsetree based}

That's why in 2014 the approach became parsetree (\emph{AST} from now on)
focused with $ocaml.4.02.00$. The AST was augmented by two meta-programming
dedicated nodes \cite{extension_nodes}\cite{attributes}. Meta-programming tools,
called PPXs, extend those nodes to other nodes:

\begin{gather*}
source \xrightarrow{\text{parsing}} AST \xrightarrow{\text{PPXs}}\\
\rightarrow expanded\ AST\xrightarrow{\text{typing}} typedtree
\end{gather*}

To extend the nodes, PPXs need to manipulate the AST data type. Given that that
data type is modified between different compiler versions, a PPX was tied to one
compiler version (is that true, Carl? Or were there other workarounds before
OMP?).

To gain cross-compiler compatibility, $ocaml\mbox{-}migrate\mbox{-}parsetree$
(OMP)\cite{omp} was invented. It converts the compiler AST version to the
version used by the PPX, then applies the PPX, and then converts back. However,
downward conversions are only well-defined if the AST doesn't contain new
features/nodes. So full cross-compiler compatibility is only given if
\begin{align*}
\text{PPX\ AST\ version} >= \text{compiler\ AST\ version}
\end{align*}
That meant that the OCaml PPX world started to lag behind the compiler releases.

Furthermore, each PPX was a separate and isolated binary; with one exception:
for type derivers, there was $ppx \_ deriving$, which would orchestrate the
different derivers.

In 2018, $ppxlib$\cite{ppxlib} was invented to solve both problems and bring
more benefits (pointed out \hyperref[benefits]{below}). It exposes one fixed
AST, which all OCaml PPXs are defined against. It coordinates with the compiler
releases to bump that AST version in parallel. That's how it has consolidated a
PPX ecosystem, which stays up-to-date.

The main problem in terms of the consolidated ecosystem is that bumping the AST
breaks a few PPXs each time. To solve that situation, there was the idea to
abstract the AST exposed by $ppxlib$ and keep that one stable. That idea was
presented at the OCaml workshop 2019\cite{workshop2019}. However, the idea
involved too much complexity and was abandoned. We now have other means to
approach the stability problem, pointed out \hyperref[stability]{below}.


\subsection{Typedtree based}

There have also been efforts to introduce typedtree based meta-programming:
$MetaOCaml$\cite{metaocaml}. It's been a very valuable experience and might
still become more some day.
%A nice feature of typedtree-based meta-programming is that the typing of the
%generated code is checked at the time of writing the meta-program. It comes at
%the cost of having to deal with two unstable representations instead of one:
%both the parsetree and the typedtree. (ToDo: I'll ask Olivier if this paragraph
%is correct and sounds good to the MetaOCaml folks)

\section{Current Situation}

The standard for meta-programming, $ppxlib$, provides very powerful machinery.
%The $ppxlib$ library allows to do general parsetree analysis, (de)-construction
%and manipulation. The most common use case is to register parsetree
%transformations via $ppxlib$'s library to $ppxlib$'s driver. The driver is one
%binary that orchestrates all registered transformations (PPXs). It's possible
%to define and register semantically quite complex transformations. However,
%it's most common and most recommended to stick to local/context-free
%transformations of one node.
However, the most common and usually most recommended workflow is to write
context-free transformations and to register those transformations via
$ppxlib$'s library to $ppxlib$'s driver. The driver is one binary that
orchestrates all registered transformations (PPXs). A context-free
transformation locally transforms one node, isolated from its semantic context.

\vspace{20pt}

(visualization)

\vspace{25pt}

Context-free transformations have three big advantages:
\begin{enumerate}

\item\hypertarget{benefits}
Among all context-free transformations, we have clear composition semantics:
$ppxlib$'s driver merges all context-free transformations into one single AST
traversal and applies each transformation successively.

\item The just mentioned merge considerably improves performance.

\item\label{responsibility}
A context-free transformation is easy to reason about and doesn't mess with its
context.

\end{enumerate}

\subsection{Hygiene}

Point \ref{responsibility}. above touches on an important topic: PPXs generate
the final representation of the program and so have an impact on how the
compiler and the editor tool $merlin$ analyze the program. That's why we define
clear hygienic rules\cite{hygiene} and try to create a culture in which PPX
authors are aware of the possible consequences.

\subsubsection*{Error handling}

It's important to handle PPX errors at the level of the AST abstraction.
Concretely, errors need to be embedded as nodes into the AST. A raising PPX
means that the compiler and $merlin$ don't receive an expanded AST.
%and won't be able to reason about the whole program.

\subsubsection*{Location invariants}

The \emph{Location} of a AST node reveals which part of the source code the node
corresponds to.
%Locations are very important for the compiler/$merlin$ to point to the right
%place when reporting errors/warnings.
There are two location invariants, which PPXs need to respect:
\begin{itemize}
\item Parent-child nodes are nested wrt. locations.
\item The locations of sibling nodes don't overlap.
\end{itemize}

\subsubsection*{Full qualification of identifiers and operators}

A PPX needs to be independent of the semantic context it's applied in.

\begin{example}[Non-qualified identifier]{example:non-qualified}
Suppose you have a PPX which injects code containing
\begin{verbatim}
compare x y
\end{verbatim}
Then the result of that comparison depends on whether $Stdlib$'s $compare$ is
shadowed inside the context the PPX is applied in or not.
\end{example}
\begin{example}[Fully qualified identifier]{example:non-qualified}
Now suppose your PPX instead fully qualifies:
\begin{verbatim}
Stdlib.compare x y
\end{verbatim}
Then the result is deterministic (we assume people don't shadow the $Stdlib$).
\end{example}

If your PPX relies on values/modules/operators that don't form part of the
$Stdlib$, you can make use of PPX runtime modules.

%\subsubsection{Quoting of injected values}
%
%(What do you think Carl and Paul-Elliot? Should we also mention quoting here?)

\subsection{Stability}\label{stability}

Additionally to being hygienic, we also want our PPX ecosystem to stay
consolidated and up-to-date. The problem: when a new compiler with its new AST
version is released and $ppxlib$ bumps its exposed AST to the new version, there
are a few PPXs that break. We have two strategies here.

\subsubsection*{Reduce the number of breakages}

There are different ways to (de-)construct the AST. A very low-level approach is
to handle the original data type directly. That workflow is clearly unstable by
the unstable AST nature. A very high-level approach is to use the PPX
$metaquot$, which comes bundled with $ppxlib$. It lets you (de-)construct nodes
by writing the corresponding OCaml syntax. That's stable due to the OCaml syntax
being backward compatible. An intermediate approach is to use helper modules
called $Ast\_builder$ and $Ast\_pattern$. That used to be unstable since the
modules are automatically generated. Good news: since $ppxlib.0.26.0 $ (2022),
we keep them manually stable!

%helper modules called $Ast\_builder$ and $Ast\_pattern$
%
%, such as:
%\begin{enumerate}
%\item\label{direct}
%Very low-level: directly handling the original data type.
%\item\label{helpers}
%More high-level: using helper modules called $Ast\_builder$ and $Ast\_pattern$.
%\item\label{metaquot}
%High-level: using the PPX $metaquot$ that comes bundled with $ppxlib$ and
%(de-)constructs nodes by writing original OCaml syntax.
%\end{enumerate}
%
%\ref{direct}. is unstable by the unstable parsetree nature. \ref{metaquot}. is stable due to
%the OCaml syntax being backward compatible. And \ref{helpers}. used to be
%unstable due to being automatically generated; however, since $ppxlib.0.26.0 $
%(2022), we keep them manually stable. When possible, we recommend
%avoiding \ref{direct}.

\subsubsection*{Patch each PPX in case of breakage}

Even with that, there are still a few PPXs that break. Therefore, when bumping
the AST, we create a big workspace containing all PPXs released on $opam$
(fulfilling a few standards), called the $ppx\_universe$\cite{universe}. We then
quickly patch the broken PPXs in the universe and open PRs. The few affected PPX
maintainers only need to review, merge and release.

%%%%%%%%%%%%%%
% References %
%%%%%%%%%%%%%%

\nocite{*}
\bibliographystyle{plain}
\bibliography{references}

\end{document}

\documentclass[12pt, a4paper, twocolumn]{article}
\usepackage{xurl}
\usepackage[super,comma,sort&compress]{natbib}
\usepackage{abstract}
\renewcommand{\abstractnamefont}{\normalfont\bfseries}
\renewcommand{\abstracttextfont}{\normalfont\small\itshape}
\usepackage{lipsum}

%%%%%%%%%%%%%%
% References %
%%%%%%%%%%%%%%

% If changing the name of the bib file, change \bibliography{test} at the bottom
\begin{filecontents}{references.bib}

@misc{camlp4,
  title        = "Link Title",
  author       = "Link Creator(s)",
  howpublished = "\url{https://example.com/}",
}

@misc{Author1,
  author       = "LastName, FirstName",
  howpublished = "\url{mailto:email@example.com}",
}

@article{ArticleReference1,
  author  = "Lastname1, Firstname1 and Lastname2, Firstname2",
  title   = "Article title",
  year    = "Year",
  journal = "Journal name",
  note    = "\url{https://dx.doi.org/...}",
}

\end{filecontents}

% Any configuration that should be done before the end of the preamble:
\usepackage{hyperref}
\hypersetup{colorlinks=true, urlcolor=blue, linkcolor=blue, citecolor=blue}

\nocite{*}
\bibliographystyle{plain}
\bibliography{test}

\end{document}
