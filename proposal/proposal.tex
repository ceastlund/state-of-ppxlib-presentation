\title{Meta-programming in OCaml\\ \large{Ppxlib: How We Got Here and\\Where We Are Now}}

\author{Paul-Elliot Anglès d'Auriac, Carl Eastlund, Sonja Heinze}

\newcommand{\abstractText}{\noindent
Nine years ago, the OCaml community let go of the "maintenance hell" of source
code meta-programming. Now meta-programming is done on the parsetree and has
evolved into a stable ecosystem. Two highlights of that evolvement: introducing
parsetree migrations to gain cross-compiler-version compatibility (OMP); and
orchestrating those migrations to create an up-to-date ecosystem with clear
composition semantics and good performance (ppxlib). In this talk, we outline
the evolution of this ecosystem its remaining challenges.
}

%%%%%%%%%%%%%%%%%
% Configuration %
%%%%%%%%%%%%%%%%%

\documentclass[10pt, a4paper, twocolumn]{article}
\usepackage{xurl}
\usepackage[super,comma,sort&compress]{natbib}
\usepackage{abstract}
\renewcommand{\abstractnamefont}{\normalfont\bfseries}
\renewcommand{\abstracttextfont}{\normalfont\small\itshape}
\usepackage{amsmath}
\usepackage[framemethod=TikZ]{mdframed}
\usepackage{hyperref}

\newcounter{example}[section]\setcounter{example}{0}
\renewcommand{\theexample}{\arabic{section}.\arabic{example}}
\newenvironment{example}[2][]{%
%\refstepcounter{example}%
\ifstrempty{#1}%
{\mdfsetup{%
frametitle={%
\tikz[baseline=(current bounding box.east),outer sep=0pt]
\node[anchor=east,rectangle,fill=blue!20]
{\strut Example~\theexample};}}
}%
{\mdfsetup{%
frametitle={%
\tikz[baseline=(current bounding box.east),outer sep=0pt]
\node[anchor=east,rectangle,fill=blue!20]
{\strut Example:~#1};}}%
}%
\mdfsetup{innertopmargin=10pt,linecolor=blue!20,%
linewidth=2pt,topline=true,%
frametitleaboveskip=\dimexpr-\ht\strutbox\relax
}
\begin{mdframed}[]\relax%
\label{#2}}{\end{mdframed}}


%%%%%%%%%%%%%%
% References %
%%%%%%%%%%%%%%

% If changing the name of the bib file, change \bibliography{test} at the bottom
\begin{filecontents}{references.bib}

\end{filecontents}

% Any configuration that should be done before the end of the preamble:
\usepackage{hyperref}
\hypersetup{colorlinks=true, urlcolor=blue, linkcolor=blue, citecolor=blue}

\begin{document}

%%%%%%%%%%%%
% Abstract %
%%%%%%%%%%%%

\twocolumn[
  \begin{@twocolumnfalse}
    \maketitle
    \begin{abstract}
      \abstractText
      \newline
      \newline
    \end{abstract}
  \end{@twocolumnfalse}
]

%%%%%%%%%%%
% Article %
%%%%%%%%%%%

\section{Introduction}
Meta-programming opens the door to boilerplate generation, conditional
compilation, domain-specific syntax extensions, and generated file inclusion. It
can be done on any representation of the program.

As a statically typed language, OCaml successively constructs two in-memory
representations of the raw representation, i.e. the source-code.

\begin{align*}
source \xrightarrow{\text{parsing}} parsetree \xrightarrow{\text{typing}} typedtree
\end{align*}

Nowadays, the clear standard for OCaml meta-programming is parsetree focused and
is done via a compiler-external package called $ppxlib$. As of today, $\sim2000$
OCaml packages out of $\sim4200$ OCaml packages on $opam$ depend on $ppxlib$.

\section{Historic overview}

On the way to that standard, we've had a long history of OCaml meta-programming
facilities.

\subsection{Source-code based}

The initial approach was source-code focused. It worked by extending the parser
to one's needs:

\begin{align*}
source \xrightarrow[\text{parsing}]{\text{extended}} parsetree \xrightarrow{\text{typing}} typedtree
\end{align*}

Among many other problems, maintaining that parallel parser with extensible
grammar was a huge maintenance burden.


\subsection{Parsetree based}

That's why in 2014 the approach became parsetree (\emph{AST} from now on)
focused with $ocaml.4.02.00$. The AST was augmented by two meta-programming
dedicated nodes. Meta-programming tools, called PPXs, extend those nodes to
other nodes:

\begin{gather*}
source \xrightarrow{\text{parsing}} AST \xrightarrow{\text{PPXs}}\\
\rightarrow expanded\ AST\xrightarrow{\text{typing}} typedtree
\end{gather*}

To extend the nodes, PPXs need to manipulate the AST data type. Changes to the
AST data type would often break PPXes that depended on a previous form.

To improve cross-compiler-version compatibility,
$ocaml\mbox{-}migrate\mbox{-}parsetree$ (OMP) was invented. It converts the
compiler AST version to the version used by the PPX, then applies the PPX, and
then converts back. Downward conversions are only well-defined if the AST
doesn't contain new features/nodes. So full cross-compiler-version compatibility
is only given if
\begin{align*}
\text{PPX\ AST\ version} >= \text{compiler\ AST\ version}
\end{align*}
That meant that the OCaml PPX world started to lag behind the compiler releases.

Initially, each PPX was a separate binary. The one early exception was $ppx \_
deriving$, which orchestrates multiple type-based derivers into a single
rewriter.

In 2018, $ppxlib$ was created, in part to solve these problems. It exposes a
single version of the AST, which is kept up to date with the latest compiler.
This consolidates the PPX ecosystem and upgrades most PPXes ``for free''.

The primary remaining problem is the few PPXes that get broken each time the AST
changes. Our initial proposal, presented at the OCaml workshop 2019, was to
create a single AST that could contain fragments from multiple versions at once.
This proved too complex and was subsequently abandoned. We now
approach \hyperref[stability]{stability} with a combination of technical and
social solutions.

\subsection{Typedtree based}

There have also been efforts to introduce typedtree based meta-programming,
especially $MetaOCaml$. MetaOCaml and PPXes are different notions of
metaprogramming with different goals and expressivity. MetaOCaml is type-safe,
staged metaprogramming that can be used to specialize values at compile-time or
run-time; it is built on a fork of the OCaml compiler. PPXes offer more general
code generation than specializing values, but lack MetaOCaml's ability to
evaluate the code and its integration with the type system.

\section{Current Situation}

PPX machinery and the $ppxlib$ library have become the standard tool for OCaml
meta-programming. They allow arbitrary AST rewriting in general, and streamlined
special cases for context-free rewriting at extension nodes as well as derived
code at types and module type definitions. The deriver links together multiple
rewriters into one binary, executing all derivers and context-free rewriting in
a single pass over an AST.

Context-free transformations have three big advantages over more general rewriting:
\begin{enumerate}

\item\hypertarget{benefits} They compose well together.

\item Running in a single pass over a file is a significant performance win.

\item\label{responsibility} They are easier to reason-about than transformations
that might affect, or be affected by, their context.

\end{enumerate}

\subsection{Hygiene}

Point \ref{responsibility}. above touches on an important topic: PPXs generate
the final representation of the program and so have an impact on how the
compiler and the editor tool $merlin$ interpret the program. Over time we have
developed hygiene guidelines for creating PPXes that ``play nicely'' within the
ecosystem.

\subsubsection*{Error handling}

Errors abort the transformation of a program. This leads to tools like $merlin$
having no output program to analyze. As a result, we have changed $ppxlib$ to
catch exceptions during rewriting, and render their message as
an \texttt{ocaml.error} node in the final AST. We encourage PPX authors to
express errors as AST nodes directly so that they can occur in the most specific
possible location, and allow the rest of rewriting to proceed.

\subsubsection*{Location invariants}

The location annotated on an AST node is used by tools like $merlin$ to associate
the final AST with source code. This can get confusing when locations are duplicated
during transformation. Our hygiene guidelines come in two parts: invariants to ensure
tools can find a unique node for a location, and tools for satifying the invariants.

Every AST node's location must contain the locations of its child nodes. Sibling
nodes' locations must not overlap. These are the invariants. To help satisfy
them, a location may be annotated as a ``ghost'' when it is a duplicate. Ghost
locations are ignored for purposes of invariants, and for tools like $merlin$
when performing a location-based lookup.

\subsubsection*{Full qualification of identifiers and operators}

A PPX needs to be independent of the semantic context it's applied in. This
property has no enforcement; it is the job of PPX authors to observe it so that
their transformations are robust.

\begin{example}[Non-qualified identifier]{example:non-qualified}
Suppose you have a PPX which injects code containing
\begin{verbatim}
compare x y
\end{verbatim}
Then the result of that comparison depends on
whether \texttt{Stdlib}'s \texttt{compare} is shadowed inside the context the
PPX is applied in or not.
\end{example}
\begin{example}[Fully qualified identifier]{example:non-qualified}
Now suppose your PPX instead fully qualifies:
\begin{verbatim}
Stdlib.compare x y
\end{verbatim}
Then the result is deterministic (we assume people don't shadow
the \texttt{Stdlib}).
\end{example}

If your PPX relies on definitions that don't form part of the \texttt{Stdlib},
the standard practice is to mint a runtime module for your PPX, give it an
explicit name such as \texttt{Ppx\_example\_runtime\_lib}.

\subsection{Stability}\label{stability}

Additionally to being hygienic, we also want our PPX ecosystem to stay
consolidated and up-to-date. The problem: when a new compiler with its new AST
version is released and $ppxlib$ bumps its exposed AST to the new version, there
are a few PPXs that break. We have two strategies here.

\subsubsection*{Reduce the number of breakages}

There are different ways to (de-)construct the AST. A very low-level approach is
to handle the original data type directly. That workflow is clearly unstable by
the unstable AST nature. A very high-level approach is to use the PPX
$metaquot$, which comes bundled with $ppxlib$. It lets you (de-)construct nodes
by writing the corresponding OCaml syntax. That's stable due to the OCaml syntax
being backward compatible. An intermediate approach is to use helper modules
called \texttt{Ast\_builder} and \texttt{Ast\_pattern}. That used to be unstable
since the modules are automatically generated. Good news: since $ppxlib.0.26.0$
(2022), we keep them manually stable!

\subsubsection*{Patch each PPX in case of breakage}

Even with that, there are still a few PPXs that break. Therefore, when bumping
the AST, we create a big workspace containing all PPXs released on $opam$
(fulfilling a few standards), called the $ppx\_universe$. We then
quickly patch the broken PPXs in the universe and open PRs. The few affected PPX
maintainers only need to review, merge and release.

\end{document}
